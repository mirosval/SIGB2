\section{Conclusion}

In the beginning we started with the point transfers and texture mapping, by detecting a person movement and overviewing an image in a video with using homography. The challenges we faced in this part were to working with matrixes and points in images and sequences. We tried to implement fast and optimized codes to get the best result and reduce the wrong results and code failure in video frames.

Next, we worked on camera calibration and some of the benefits of having a calibrated camera like augmented reality. The biggest challenge was to understand in what order should we apply the homographies to the points to draw them on the screen.

We had a successful achievement on working with homographies and its usage. Texture mapping and augmentation were really excitement and useful and surely in the future more developers are going to use these technics in their applications.
